\documentclass[]{article}

\begin{document}
\section{Bijlage XML Blokken}
\label{bijlagexml}
\subsection{Variables en Type}

\subsubsection{Variabele blok}
De \texttt{makeVar} blok heeft een bepaalde unieke naam.	
Dit element bevat een type element.
\lstset{language=XML}
\begin{lstlisting}
<makeVar name="name of variable" type="type" x="0" y="0" />
\end{lstlisting}
De DOCTYPE declaration: 
\lstset{language=XML}
\begin{lstlisting}
<!ELEMENT makeVar (type)>
<!ATTLIST makeVar name CDATA #REQUIRED type CDATA #REQUIRED x CDATA #IMPLIED y CDATA #IMPLIED >
\end{lstlisting}

\subsubsection{Value}
De \texttt{value} blok bevat een string die de data voorstelt. 
\lstset{language=XML}
\begin{lstlisting}
<value> value </value>
\end{lstlisting}
De DOCTYPE declaration: 
\lstset{language=XML}
\begin{lstlisting}
<!ATTLIST value x CDATA #IMPLIED y CDATA #IMPLIED>
<!ELEMENT value (#PCDATA)>
\end{lstlisting}

\subsubsection{Set blok}
De \texttt{setVar} blok bevat een variabele en een value die geassigned zal worden aan deze variabele.
\lstset{language=XML}
\begin{lstlisting}
<setVar name="name">
	<value> value </value>
</setVar>
\end{lstlisting}
De DOCTYPE declaration: 
\lstset{language=XML}
\begin{lstlisting}
<!ATTLIST setVar name CDATA #IMPLIED x CDATA #IMPLIED y CDATA #IMPLIED>
<!ELEMENT setVar (var|value|concat|length|charat|random|operator)>
\end{lstlisting}
\subsubsection{Print blok}
De \texttt{print} blok bevat een block die uitgeprint zal worden naar de console.
\lstset{language=XML}
\begin{lstlisting}
<print>
	<value> value </value>
</print>
\end{lstlisting}
De DOCTYPE declaration: 
\lstset{language=XML}
\begin{lstlisting}
<!ATTLIST print x CDATA #IMPLIED y CDATA #IMPLIED>
<!ELEMENT print (var|value|concat|length|charat|random|operator)>
\end{lstlisting}
\subsubsection{var blok}
De \texttt{var} blok heeft een naam waarmee een variable mee kan worden aangesproken of gemanipuleerd.
\lstset{language=XML}
\begin{lstlisting}
<var name="varName"/>
\end{lstlisting}
De DOCTYPE declaration: 
\lstset{language=XML}
\begin{lstlisting}
<!ELEMENT var EMPTY>
<!ATTLIST var name CDATA #REQUIRED>
\end{lstlisting}

\subsubsection{null blok}
Wordt gebruikt om in bepaalde situaties "geen waarde" aan te duiden.
\lstset{language=XML}
\begin{lstlisting}
<null/>
\end{lstlisting}
De DOCTYPE declaration: 
\lstset{language=XML}
\begin{lstlisting}
<!ELEMENT null EMPTY>
\end{lstlisting}

\subsection{Locks}
\subsubsection{Lock}
De \texttt{lock} blok bevat een variabele die gelocked moet worden.	
\lstset{language=XML}
\begin{lstlisting}
<lock name="" />
\end{lstlisting}
De DOCTYPE declaration: 
\lstset{language=XML}
\begin{lstlisting}
<!ATTLIST lock name CDATA #IMPLIED x CDATA #IMPLIED y CDATA #IMPLIED>
\end{lstlisting}

\subsubsection{Unlock}
De \texttt{unlock} blok bevat een variabele die geunlocked moet worden.	
\lstset{language=XML}
\begin{lstlisting}
<unlock name="name" />
\end{lstlisting}
De DOCTYPE declaration: 
\lstset{language=XML}
\begin{lstlisting}
<!ATTLIST lock name CDATA #IMPLIED x CDATA #IMPLIED y CDATA #IMPLIED>
\end{lstlisting}

\subsection{Control Blokken}
\subsubsection{Forever block}
De \texttt{forever} block bevat een block code dat wordt uitgevoerd.
\lstset{language=XML}
\begin{lstlisting}
<forever>
  <block> code </block>
</forever>
\end{lstlisting}
De DOCTYPE declaration: 
\lstset{language=XML}
\begin{lstlisting}
<!ATTLIST forever x CDATA #IMPLIED y CDATA #IMPLIED>
<!ELEMENT forever (block)>
\end{lstlisting}
\subsubsection{If blok}
De \texttt{if} blok bevat een conditie en een code blok.
\lstset{language=XML}
\begin{lstlisting}
<if>
  <cond> condition </cond>
  <block> code </block>
</if>
\end{lstlisting}
De DOCTYPE declaration: 
\lstset{language=XML}
\begin{lstlisting}
<!ATTLIST if x CDATA #IMPLIED y CDATA #IMPLIED>
<!ELEMENT if (cond,block)>
\end{lstlisting}
\subsubsection{If-else blok}
De \texttt{if-else} blok bevat een conditie en twee code blokken.
\lstset{language=XML}
\begin{lstlisting}
<if-else>
  <cond> condition </cond>
  <block> if-code </block>
  <block> else-code </block>
</if-else>
\end{lstlisting}
De DOCTYPE declaration: 
\lstset{language=XML}
\begin{lstlisting}
<!ATTLIST if-else x CDATA #IMPLIED y CDATA #IMPLIED>
<!ELEMENT if-else (cond,block,block)>
\end{lstlisting}
\subsubsection{Conditie blok}
De \texttt{conditie} bevat een variable of een logische expressie.
\lstset{language=XML}
\begin{lstlisting}
<cond>
  <var name="varName"/>
</cond>
\end{lstlisting}
De DOCTYPE declaration: 
\lstset{language=XML}
\begin{lstlisting}
<!ELEMENT  cond (var|value|concat|length|charat|random|operator)>
\end{lstlisting}
\subsubsection{While blok}
De \texttt{while} bevat een conditie en een code blok.
\lstset{language=XML}
\begin{lstlisting}
<while>
  <cond> condition </cond>
  <block> code </block>
</while>
\end{lstlisting}
De DOCTYPE declaration: 
\lstset{language=XML}
\begin{lstlisting}
<!ATTLIST while x CDATA #IMPLIED y CDATA #IMPLIED>
<!ELEMENT while (cond,code)>
\end{lstlisting}

\subsection{Physics}
\subsubsection{Move blok}
De \texttt{move} blok stelt een translatie voor.	
\lstset{language=XML}
\begin{lstlisting}
<move x="0.0" y="0.0">
    <x/>
    <y/>
</move>
\end{lstlisting}
De DOCTYPE declaration: 
\lstset{language=XML}
\begin{lstlisting}
<!ELEMENT move (x,y)>
<!ATTLIST move x CDATA #IMPLIED y CDATA #IMPLIED>
\end{lstlisting}

\subsubsection{Show}
De \texttt{show} blok maakt een instance zichtbaar of doet niets indien de instance al zichtbaar was.
\lstset{language=XML}
\begin{lstlisting}
<show />
\end{lstlisting}
De DOCTYPE declaration: 
\lstset{language=XML}
\begin{lstlisting}
<!ATTLIST show x CDATA #IMPLIED y CDATA #IMPLIED>
<!ELEMENT show EMPTY>
\end{lstlisting}

\subsubsection{Hide}
De \texttt{hide} blok maakt een instance onzichtbaar of doet niets indien de instance al onzichtbaar was.
\lstset{language=XML}
\begin{lstlisting}
<hide />
\end{lstlisting}
De DOCTYPE declaration: 
\lstset{language=XML}
\begin{lstlisting}
<!ATTLIST hide x CDATA #IMPLIED y CDATA #IMPLIED>
<!ELEMENT hide EMPTY>
\end{lstlisting}

\subsubsection{Change appearance}
De \texttt{changeAppearance} blok maakt het mogelijk voor een instance om zijn uiterlijk te veranderen. De id is de ID van zijn nieuw uiterlijk.
\lstset{language=XML}
\begin{lstlisting}
<changeAppearance id="0"/>
\end{lstlisting}
De DOCTYPE declaration: 
\lstset{language=XML}
\begin{lstlisting}
<!ELEMENT appear EMPTY>
<!ATTLIST hide x CDATA #IMPLIED y CDATA #IMPLIED>
\end{lstlisting}

\subsection{String operators}
\subsubsection{Concat}
De \texttt{concat} blok laat toe om 2 strings samen te voegen.
\lstset{language=XML}
\begin{lstlisting}
<concat>
	<null>
	<var name="right var to concat">
<concat>
\end{lstlisting}
De DOCTYPE declaration: 
\lstset{language=XML}
\begin{lstlisting}
<!ATTLIST concat x CDATA #IMPLIED y CDATA #IMPLIED>
<!ELEMENT concat (var|value|concat|length|charat|random|null|operator, var|value|concat|length|charat|random|operator)>
\end{lstlisting}

\subsubsection{Length}
De \texttt{strlen} blok laat toe om te lengte van een string op te vragen.
\lstset{language=XML}
\begin{lstlisting}
<length>
	<var name="string">
</length>
\end{lstlisting}
De DOCTYPE declaration: 
\lstset{language=XML}
\begin{lstlisting}
<!ATTLIST length x CDATA #IMPLIED y CDATA #IMPLIED>
<!ELEMENT length (var|value|concat|length|charat|random|operator)>
\end{lstlisting}

\subsubsection{CharAt}
De \texttt{charAt} blok laat toe om character op een bepaalde index op te vragen.
\lstset{language=XML}
\begin{lstlisting}
<charat>
	<value> index </value>
	<var name="string"/>
<charat>
\end{lstlisting}
De DOCTYPE declaration: 
\lstset{language=XML}
\begin{lstlisting}
<!ATTLIST charat x CDATA #IMPLIED y CDATA #IMPLIED>
<!ELEMENT charat (var|value|concat|length|charat|random|null|operator, var|value|concat|length|charat|random|operator)>
\end{lstlisting}

\subsection{Trivia}
\subsubsection{Sleep blok}
De \texttt{print} blok bevat een block die bepaald hoeveel milliseconden er geslapen moet worden.
\lstset{language=XML}
\begin{lstlisting}
<sleep>
	<value> value </value>
</sleep>
\end{lstlisting}
De DOCTYPE declaration: 
\lstset{language=XML}
\begin{lstlisting}
<!ATTLIST sleep x CDATA #IMPLIED y CDATA #IMPLIED>
<!ELEMENT sleep (var|value|concat|length|charat|random|operator)>
\end{lstlisting}

\subsection{Operators}
\subsubsection{Random}
De \texttt{random} blok stelt een functie voor die een random gekozen getal teruggeeft tussen de meegegeven bounds. 
\lstset{language=XML}
\begin{lstlisting}
<random/>
\end{lstlisting}
De DOCTYPE declaration: 
\lstset{language=XML}
\begin{lstlisting}
<!ATTLIST random x CDATA #IMPLIED y CDATA #IMPLIED>
<!ELEMENT random EMPTY>
\end{lstlisting}\
\subsubsection{Operator}
De \texttt{operator} blok stelt een operator voor zoals +,-,/,*,$<$,$>$,... . Dit gebeurt met twee operatoren. Als de tweede kind-element \texttt{dummy} is, dan is de operator een unaire operator.
\lstset{language=XML}
\begin{lstlisting}
<operator op="+">
    <NULL/> 
    <dummy/>
</operator>
\end{lstlisting}
De DOCTYPE declaration: 
\lstset{language=XML}
\begin{lstlisting}
<!ELEMENT operator (NULL,|var|value|concat|length|charat|random|operator , NULL|dummy|var|value|concat|length|charat|random|operator)>
<!ATTLIST operator op CDATA #REQUIRED x CDATA #IMPLIED y CDATA #IMPLIED>
\end{lstlisting}

\subsection{Functions en Handlers}
\subsubsection{Handler}
De \texttt{handler} blok stelt een speciale functie voor die een event opvangt. 
\lstset{language=XML}
\begin{lstlisting}
<handler name="name" event="type of event">
	<block>
		code
	</block>
</handler>
\end{lstlisting}
De DOCTYPE declaration: 
\lstset{language=XML}
\begin{lstlisting}
<!ELEMENT handler (block)>
<!ATTLIST handler name CDATA #required event CDATA #IMPLIED x CDATA #IMPLIED y CDATA #IMPLIED>
\end{lstlisting}

\subsubsection{Param}
De \texttt{param} blok stelt een parameter voor van een functie, deze heeft een type en een identifier. 
\lstset{language=XML}
\begin{lstlisting}
<param type="string" name="name1"/>

\end{lstlisting}
De DOCTYPE declaration: 
\lstset{language=XML}
\begin{lstlisting}
<!ELEMENT param EMPTY>
<!ATTLIST param type CDATA #REQUIRED name CDATA #REQUIRED>
\end{lstlisting}

\subsubsection{Return function}
Het \texttt{return} element stelt een return voor van een functie, deze heeft een type. 
\lstset{language=XML}
\begin{lstlisting}
<return type="string"/>
\end{lstlisting}
De DOCTYPE declaration: 
\lstset{language=XML}
\begin{lstlisting}
<!ELEMENT return EMPTY>
<!ATTLIST return type CDATA #REQUIRED>
\end{lstlisting}

\subsubsection{Params}
De \texttt{params} blok is een collectie voor alle parameters van een class.
\lstset{language=XML}
\begin{lstlisting}
<params>
	<param type="string" name="name1"/>
</params>
\end{lstlisting}
De DOCTYPE declaration: 
\lstset{language=XML}
\begin{lstlisting}
<!ELEMENT params (param)*>
\end{lstlisting}

\subsubsection{Function}
De \texttt{function} blok stelt een functie voor die opgeroepen kan worden in een ander stuk code van deze class. 
\lstset{language=XML}
\begin{lstlisting}
<function name="name">
    <return type="NULL"/>     
     <params>
	    <param type="string" name="name1"/>
	    <param type="string" name="name2"/>
    </params>
	<block>
		code
	</block>
</function>
\end{lstlisting}
De DOCTYPE declaration: 
\lstset{language=XML}
\begin{lstlisting}
<!ELEMENT function (param*, block)>
<!ATTLIST function name CDATA #REQUIRED x CDATA #IMPLIED y CDATA #IMPLIED>
\end{lstlisting}
\subsubsection{Return Blok}
Een \texttt{return} blok bevat variabelen die hij returned. Volgorde is hier van belang.
\begin{lstlisting}
\lstset{language=XML}
<return>
	<var name="varName" />
</return>	
\end{lstlisting}
De DOCTYPE declaration: 
\lstset{language=XML}
\begin{lstlisting}
<!ATTLIST return x CDATA #IMPLIED y CDATA #IMPLIED>
<!ELEMENT return (var)*>
\end{lstlisting}
\subsubsection{FunctionCall blok}
Een \texttt{FunctionCall} blok stelt een functie oproep voor en bevat variabelen voor de oproep.
De laaste variable is de return waarde als de functie iets returned.
\lstset{language=XML}
\begin{lstlisting}
<functionCall name="functioName">
  <params>
      <var name="varName1"/>
      <var name="varName2"/>
  </params>
  <returns/>
</functionCall>
\end{lstlisting}
De DOCTYPE declaration: 
\lstset{language=XML}
\begin{lstlisting}
<!ELEMENT functionCall (params, returns)>
<!ATTLIST functionCall name CDATA #REQUIRED x CDATA #IMPLIED y CDATA #IMPLIED>


<!ELEMENT returns (var)*>
<!ELEMENT params (var)*>
\end{lstlisting}

\subsection{Block}
De \texttt{block} blok stelt een groepering van code voor. 
\lstset{language=XML}
\begin{lstlisting}
<block>
	code
</block>
\end{lstlisting}
De DOCTYPE declaration: 
\lstset{language=XML}
\begin{lstlisting}
<!ELEMENT block (makeVar|setVar|move|show|hide|appear|if|if-else|sleep|
			while|forever|emit|functionCall|print|lock|unlock|return)*>
<!ATTLIST block name CDATA #REQUIRED>
\end{lstlisting}

\subsection{Class}
\subsubsection{InputEvent}
De \texttt{inputEvent} blok is een binnenkomende event van een class.
\lstset{language=XML}
\begin{lstlisting}
<inputEvent type="ev1"/>
\end{lstlisting}
De DOCTYPE declaration: 
\lstset{language=XML}
\begin{lstlisting}
<!ELEMENT inputEvent EMPTY>
<!ATTLIST inputEvent type CDATA #REQUIRED>
\end{lstlisting}

\subsubsection{OutputEvent}
De \texttt{outputEvent} blok is een uitgaande event van een class.
\lstset{language=XML}
\begin{lstlisting}
<outputEvent type="ev2"/>
\end{lstlisting}
De DOCTYPE declaration: 
\lstset{language=XML}
\begin{lstlisting}
<!ELEMENT outputEvent EMPTY>
<!ATTLIST outputEvent type CDATA #REQUIRED>
\end{lstlisting}

\subsubsection{Costume}
De \texttt{costumes} blok is een collectie voor alle costumes.
\lstset{language=XML}
\begin{lstlisting}
<costume name="s" path="path"/>
\end{lstlisting}
De DOCTYPE declaration: 
\lstset{language=XML}
\begin{lstlisting}
<!ELEMENT outputEvent EMPTY>
<!ATTLIST outputEvent name CDATA #REQUIRED path CDATA #REQUIRED>
\end{lstlisting}


\subsubsection{Events}
De \texttt{events} blok is een collectie voor alle input events van een class.
\lstset{language=XML}
\begin{lstlisting}
<events>
	<inputEvent type="ev1"/>
</events>
\end{lstlisting}
De DOCTYPE declaration: 
\lstset{language=XML}
\begin{lstlisting}
<!ELEMENT events (inputEvent)*>
\end{lstlisting}

\subsubsection{Emits}
De \texttt{emits} blok is een collectie voor alle emits die een class kan doen.
\lstset{language=XML}
\begin{lstlisting}
<emits>
	<outputEvent type="ev2"/>
</emits>
\end{lstlisting}
De DOCTYPE declaration: 
\lstset{language=XML}
\begin{lstlisting}
<!ELEMENT emits (outputEvent)*>
\end{lstlisting}

\subsubsection{Handlers}
De \texttt{handlers} blok is een collectie voor alle handlers van een class.
\lstset{language=XML}
\begin{lstlisting}
<handlers>
	<handler name="hand" event="ev1">
		code
	<\handler>
</handlers>
\end{lstlisting}
De DOCTYPE declaration: 
\lstset{language=XML}
\begin{lstlisting}
<!ELEMENT handlers (handler)*>
\end{lstlisting}

\subsubsection{Functions}
De \texttt{functions} blok is een collectie voor alle functions van een class.
\lstset{language=XML}
\begin{lstlisting}
<functions>
	<function name="func">
		code
	<\function>
</functions>
\end{lstlisting}
De DOCTYPE declaration: 
\lstset{language=XML}
\begin{lstlisting}
<!ELEMENT functions (function)*>
\end{lstlisting}

\subsubsection{MemberVariables}
De \texttt{memberVariables} blok is een collectie voor alle member variables van een class.
\lstset{language=XML}
\begin{lstlisting}
<memberVariables>
	<member type="number" name="var1" />
</memberVariables>
\end{lstlisting}
De DOCTYPE declaration: 
\lstset{language=XML}
\begin{lstlisting}
<!ELEMENT memberVariables (member)*>
\end{lstlisting}

\subsubsection{Floating blocks}
De \texttt{floatingBlocks} blok is een collectie voor alle losstaande blokken.
\lstset{language=XML}
\begin{lstlisting}
<floatingBlocks>
	<blocka />
</floatingBlocks>
\end{lstlisting}
De DOCTYPE declaration: 
\lstset{language=XML}
\begin{lstlisting}
<!ELEMENT floatingBlocks (block)*>
\end{lstlisting}

\subsubsection{Costumes}
De \texttt{costumes} blok is een collectie voor alle costumes.
\lstset{language=XML}
\begin{lstlisting}
<costumes>
	<costume name="s" path="path"/>
</costumes>
\end{lstlisting}
De DOCTYPE declaration: 
\lstset{language=XML}
\begin{lstlisting}
<!ELEMENT costumes (costume)*>
\end{lstlisting}

\subsubsection{Class}
De \texttt{class} blok stelt volledige class voor. Deze list al zijn input events, alle verschillende emits, alle member variabelen, alle handler functions, alle floating blocks en alle costumes en alle gewone functions.
\lstset{language=XML}
\begin{lstlisting}
<class name="name">
	<events>
		<inputEvent type="ev1/">
	</events>
	<emits>
		<outputEvent type="ev2"/>
	</emits>
	<handlers>
		<handler name="hand" event="ev1">
			code
		<\handler>
	</handlers>
	<functions>
		<function name="func">
			code
		<\function>
	</functions>
	<memberVariables>
		<member type="number" name="var1" />
	</memberVariables>
	<costumes/>
	<floatingBlocks>
	    <block/>
	</floatingBlocks>
</class>
\end{lstlisting}
De DOCTYPE declaration: 
\lstset{language=XML}
\begin{lstlisting}
<!ELEMENT class (events, emits, handlers, functions, memberVariables, costumes, floatingBlocks)>
<!ATTLIST class name CDATA #REQUIRED>
\end{lstlisting}

\subsection{Events en Emits}
\subsubsection{Emit blok}
Het \texttt{emit} block bevat de naam event en de members van van de message van dit event.
De members zijn variabelen, ze hebben een extra attribuut. Namelijk om de members te matchen.
\lstset{language=XML}
\begin{lstlisting}
<emit eventName="event">
  <var member="a" name="var1">
  <var member="b" name"var2">
</emit>
\end{lstlisting}
De DOCTYPE declaration: 
\lstset{language=XML}
\begin{lstlisting}
<!ELEMENT emit (var)*>
<!ATTLIST emit eventName CDATA #REQUIRED x CDATA #IMPLIED y CDATA #IMPLIED>
\end{lstlisting}
\subsubsection{Event blok}
Een \texttt{event} blok voor het tonen en cree\"{e}ren van een event. Deze bevat een uniek type en members van een specifiek type met een unieke naam in het event.
\lstset{language=XML}
\begin{lstlisting}
<event type="eventName">
  <member type="memberType" name"memberName"/>
  <member type="memberType2" name"memberName2"/>
</event>
\end{lstlisting}
De DOCTYPE declaration: 
\lstset{language=XML}
\begin{lstlisting}
<!ELEMENT event (member)*>
<!ATTLIST event type CDATA #REQUIRED>
\end{lstlisting}
\subsubsection{Member blok}
De \texttt{member} blok kan een een event zitten. Dit is een variabele dat een type heeft en een naam.
\lstset{language=XML}
\begin{lstlisting}
<member type="memberType" name="memberName"/>
\end{lstlisting}
De DOCTYPE declaration: 
\lstset{language=XML}
\begin{lstlisting}
<!ELEMENT member EMPTY>
<!ATTLIST member type CDATA #REQUIRED name CDATA #REQUIRED>
\end{lstlisting}
\subsubsection{Access blok}
De \texttt{access} blok bevat de naam van de member die men wilt aanspreken.
Deze geeft deze variable zijn value terug.
\lstset{language=XML}
\begin{lstlisting}
<access name="memberName"/>
\end{lstlisting}
De DOCTYPE declaration: 
\lstset{language=XML}
\begin{lstlisting}
<!ELEMENT access EMPTY>
<!ATTLIST access name CDATA #REQUIRED >
\end{lstlisting}

\subsection{Instances en Wires}
\subsubsection{Instance }
Dit stelt de XML voor een instance op te slaan voor. Een instance heeft een class waarbij het behoort een positie en een unieke naam.
\lstset{language=XML}
\begin{lstlisting}
<instance name="instanceName" class="className" x="X" y="Y" />
\end{lstlisting}
De DOCTYPE declaration: 
\lstset{language=XML}
\begin{lstlisting}
<!ELEMENT instance EMPTY>
<!ATTLIST instance event CDATA #REQUIRED name CDATA #REQUIRED sprite			 CDATA #REQUIRED  x CDATA #REQUIRED y CDATA #REQUIRED>
\end{lstlisting}
\subsubsection{Wire}
Een wire heeft twee instances en het event dat er tussen verstuurd wordt.
\lstset{language=XML}
\begin{lstlisting}
<instance from="instanceName" to="instanceName2" event="eventName" />
\end{lstlisting}
De DOCTYPE declaration: 
\lstset{language=XML}
\begin{lstlisting}
<!ELEMENT wire EMPTY>
<!ATTLIST wire from CDATA #REQUIRED to CDATA #REQUIRED event CDATA #REQUIRED >
\end{lstlisting}
\subsubsection{WireFrame}
Een wireFrame bevat instances en de wires tussen die instances.
\begin{lstlisting}
\lstset{language=XML}
<wireFrame>
     <instances>
	    <instance name="instance1" class="className" x="1" y="1"/>
	    <instance name="instance2" class="className2" x="1" y="1"/>
	</instances>
	<wires>
	    <wire from="instance1" to="intance2" event="eventName"/>
	</wires>
</wireFrame>	
\end{lstlisting}
De DOCTYPE declaration: 
\lstset{language=XML}
\begin{lstlisting}
<!ELEMENT wireFrame (instance|wire)*>
\end{lstlisting}
\end{document}