\documentclass[]{article}

\begin{document}

\section{Taakverdeling en Planning}
\label{Taakverdeling}

\subsection{Planning}
We werken iteratief aan het software project. Er wordt gezorgd dat het eindverslag wekelijks bijgewerkt wordt. Voor de verschillende modules wordt eerst de kern ge\"{i}mplementeerd zodanig dat er een basis versie van het project bestaat. Deze versie word in verdere weken uitgewerkt tot een finale versie. In deze lijst staan telkens de belangrijke elementen die die week ge\"{i}mplementeerd worden.
\begin{itemize}
\item Week 1: Exceptions en Variables module maken. Aanmaak van enkele blokken uit de Blocks Module om te kunnen testen. Beginnen aan het aanmaken van de Core module
\item Week 2: Aanmaken van de Core module zodanig dat de Virtual machine operationeel is. Aanmaak XML inlezing zodat programma's al gemaakt kunnen worden in XML formaat. Hierdoor kan de Core module al volledig getest worden.
\item Week 3: Aanmaken Collections en Runtime module. Hierdoor ligt een basis voor de GUI klaar 
\item Week 4: Aanmaken van een eerste GUI prototype (en de multilanguage klasse) en de blok modellen.
\item 30 April 2015: Tussentijdse rapportering met opdrachtgevers.
\item Week 5: Verdere implementatie van alle blokken.
\item Week 6: Professioneel look geven aan de GUI.
\item Week 7: GUI verder afwerken en de blokken en hun modellen volledig afwerken.
\item Week 8: Extra tijd om extra's te implementeren of eventuele achterstand in te halen.
\item Week 9: Extra tijd om extra's te implementeren of eventuele achterstand in te halen.
\item 4 Juni 2015: Inleveren eindverslag en project.
\item 9 Juni 2015: Eindpresentatie finale software project.
\end{itemize}

\subsection{Taakverdeling}
Hierin staat beschreven wie welke onderdelen van een bepaalde module implementeerd. 
\begin{itemize}
\item Exceptions module: Matthijs implementeerd alle exceptions
\item Variables module: Axel implementeerd deze module volledig. 
\item File module: Language klasse wordt gemaakt door Axel
\item File module: Aangezien de dataparser klasse zeer groot is (zo'n 40 functies) zal ieder van ons de helft van deze functies implementeren. Deze functies zijn gericht op implementatie (een functie kan een while loop inladen of opslaan). Hierdoor is er nog geen exacte opsplitsen voor wie welke functie implementeert.
\item Core module: Axel implementeerd de Instance, Class, Proces en FunctionFrame. Matthijs maakt de Virtual Machine, Event en de Event Dispatcher.
\item Collections module: ClassPool en EventPool klassen worden gemaakt door Axel. De WireInstance en het WireFrame worden gemaakt door Matthijs. 
\item Runtime module: Matthijs maakt de Runtime. Aangezien de compiler klasse zeer groot is (zo'n 40 functies) zal ieder van ons de helft van deze functies implementeren. Deze functies zijn gericht op implementatie (een functie kan een while loop compiler bv). Hierdoor is er nog geen exacte opsplitsen voor wie welke functie implementeert.
\item Block module: Aangezien deze module zeer groot is (zo'n 40 klassen) zal ieder van ons de helft van deze klassen implementeren. Deze klassen zijn gericht op implementatie (een klasse kan een while loop voorstellen). Hierdoor is er nog geen exacte opsplitsen voor wie welke klasse implementeert.
\item Model module: Aangezien deze module zeer groot is (zo'n 40 klassen) zal ieder van ons de helft van deze klassen implementeren. Deze klassen zijn gericht op implementatie (een klasse kan een while loop voorstellen). Hierdoor is er nog geen exacte opsplitsen voor wie welke klasse implementeert.
\item Drag en drop: Dit wordt eerst samen bekeken. Zodat we goed weten hoe dit ge\"{i}mplementeerd moet worden.
\item WireFrame view: Matthijs implementeerd de WireFrame view (en andere benodigde views hierbij).
 
\item Event view + menu balk: Axel implementeerd de event view (en andere benodigde views hierbij) alsook de menu balk.

\item Programmeer view: Dit gaat opnieuw een grote hoeveelheid klassen zijn (voor de verschillende views van de verschillende blokken). Aangezien dit veel werk is, worden deze klassen opgesplitst. 
\end{itemize}

\end{document}